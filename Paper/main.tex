%% Short data paper template
%% Created by Simon Hengchen and Nilo Pedrazzini for the Journal of Open Humanities Data (https://openhumanitiesdata.metajnl.com)

\documentclass{article}
\usepackage[english]{babel}
\usepackage[utf8]{inputenc}
\usepackage{amssymb}
\usepackage{amsmath}
\usepackage{amsthm}
\usepackage{johd}
\usepackage{braket}
\usepackage{multicol}
\usepackage{todonotes}

\newtheorem{theorem}{Theorem}
\newtheorem{lemma}[theorem]{Lemma}
\newtheorem{definition}[theorem]{Definition}
\newtheorem{proposition}[theorem]{Proposition}
\newtheorem{corollary}[theorem]{Corollary}
\newtheorem{thesis}[theorem]{Thesis}

\newcommand{\N}{\mathbb{N}}
\newcommand{\R}{\mathbb{R}}
\newcommand{\C}{\mathbb{C}}
\newcommand{\D}{\mathbb{D}}
\newcommand{\DC}{\mathbb{DC}}
\newcommand{\GD}{\mathbb{GD}}
\newcommand{\GDC}{\mathbb{GDC}}
\newcommand{\Z}{\mathcal{O}_\e}
\newcommand{\e}{\epsilon}

\newcommand{\E}{\mathcal{E}}
\newcommand{\F}{\mathcal{F}}
\newcommand{\G}{\mathcal{G}}
\newcommand{\til}{\widetilde}
\renewcommand{\bar}{\overline}

\renewcommand{\Re}{\operatorname{Re}}
\renewcommand{\Im}{\operatorname{Im}}
\newcommand{\Sig}{\operatorname{Sig}}
\newcommand{\Inf}{\operatorname{Inf}}

\newcommand\dstate[2]{\frac{\mathrm{d}\ket{#1}}{\mathrm{d}#2}}

\newcommand\ketbra[2]{\ket{#1}\!\bra{#2}}

\newcommand{\norm}[1]{\left\lVert#1\right\rVert}

\title{Postulates of quantum mechanics with dual-complex numbers}

\author{Nathan Houyet$^{a}$, Dogukan Bakircioglu$^{b}$, Pablo Arrighi$^{b}$ \\
        \small $^{a}$Université Paris-Saclay (student), France \\
        \small $^{b}$QuaCS, LMF, France \\
}
\date{}

\begin{document}
\maketitle

\section{Introduction}
\section{Dual numbers}
\noindent Extending real numbers with an imaginary unit $i$ with $i^2 = -1$ gives rise to the complex numbers. Complex numbers form a field and are used in quantum mechanics to represent physical systems and their evolution using complex linear algebra. The set of complex numbers is $\C = \{a + bi \; | \; a, b \in \R \text{ and } i^2 = 0\}$. If a number $z \in \C$ has the form $z = a + bi$ we denote $a$ the real part and $b$ the imaginary part. We can define the functions $\Re$ and $\Im$ as follows
\begin{multicols}{3}
\noindent
\begin{equation}
\Re: \C \to \R: a + bi \to a
\end{equation}

\columnbreak
\noindent
\begin{equation}
\Im: \C \to \R: a + bi \to b
\end{equation}

\columnbreak
\noindent
\begin{equation}
\forall z \in \C, z = \Re(z) + \Im(z)i
\end{equation}

\end{multicols}

Dual numbers $\D$ were first introduced by \cite{clifford1871} and follow a similar idea. They are generated by introducing an imaginary unit $\e$ with $\e^2 = 0$ to the reals. $\e$ is commuting in multiplication and addition with reals. To avoid confusion with complex, we call the real part of a dual the significant part ($\Sig$) and its imaginary part the infinitesimal part ($\Inf$).

\begin{multicols}{3}
\noindent
\begin{equation}
\Sig: \D \to \R: a + b\e \to a
\end{equation}

\columnbreak
\noindent
\begin{equation}
\Inf: \D \to \R: a + b\e \to b
\end{equation}

\columnbreak
\noindent
\begin{equation}
\forall d \in \D, d = \Sig(d) + \Inf(d)\e
\end{equation}

\end{multicols}

Extending the real numbers with both $i$ and $\e$ gives rise to dual-complex numbers $\DC$. We can see the complex, dual and dual-complex numbers as quotient rings:

\begin{multicols}{3}

\noindent
\begin{equation}
\C \approx \R [i]/\langle i^2+1 \rangle
\end{equation}

\columnbreak

\noindent
\begin{equation}
\D \approx \R [\e]/\langle \e^2 \rangle
\end{equation}

\columnbreak

\noindent
\begin{equation}
\DC \approx \C [\e]/\langle \e^2 \rangle
\end{equation}
\end{multicols}

Alternatively, dual-complex (resp. dual) numbers can be seen as the subalgebra of the matrices of the form

\begin{equation}
\begin{pmatrix}
 a & b\\
 0 & a
\end{pmatrix}
\end{equation}

Where $a, b \in \C$ (resp. $a, b \in \R$). Dual and dual-complex numbers are used in automatic differentiation as they allow to ``cut'' the Taylor expansion from the second term. \cite{baydin2018, rev2016}

\begin{equation}
f(z + \e) = f(z_0) + \e f'(z_0) + \e^2 f''(z_0)/2 + \dots = f(z_0) + \e f'(z_0).
\end{equation}

Higher-order versions of dual-numbers were developed to deal with higher-order derivative. See \cite{szi2021, behr2019}.

\subsubsection*{Ring operations and inversion on dual-complex numbers}

Addition and multiplication are defined as follows:

\begin{equation}
(z_1 + t_1 \e) + (z_2 + t_2 \e) = (z_1 + z_2) + (t_1 + t_2) \e
\end{equation}

\noindent \begin{equation}
(z_1 + t_1 \e) (z_2 + t_2 \e) = (z_1 z_2) + (t_1 z_2 + z_1 t_2) \e
\end{equation}

Notice how with the infinitesimal unit $z_2 t_2 \e^2$ vanish into oblivion when with the imaginary unit it becomes real and negative.

A dual-complex is called infinitesimal when its significant part is zero. The set of infinitesimal dual-complex numbers is denoted $\Z = \{0 + t \e | t \in \C\} = \e \C$. A number that is not infinitesimal is called appreciable. We can observe three cases when dividing $w_1 = z_1 + t_1 \e$ by $w_2 = z_2 + t_2 \e$:

\begin{enumerate}
        \item When $w_2$ is appreciable, i.e.\ $z_2 \neq 0$. In this case, the inverse of $w_2$ exists and division is defined as
        \begin{equation}
        \frac{z_1 + t_1 \e}{z_2 + t_2 \e} = \frac{(z_1 + t_1 \e)(z_2 - t_2 \e)}{(z_2 + t_2 \e)(z_2 - t_2 \e)} = \frac{z_1 z_2 - z_1 t_2 \e + t_1 z_2 \e}{z_2^2} = \frac{z_1}{z_2} + \frac{z_2 t_1 - z_1 t_2 }{z_2^2} \e
        \end{equation}
        \item $z_2 = 0$ and we divide the infinitesimal part ($z_1 = 0$). The usual constraint that $w' = \frac{w_1}{w_2} \iff w'w_2 = w1$ leaves us (infinitely) many solutions. Let's say that we want to find $w' = z' + t'$, the quotient of $w_1$ and $w_2$.
        \begin{equation}
        z' + t' \e = \frac{t_1 \e}{t_2 \e} \iff z' t_2 \e + t' t_2 \e^2 = t_1 \e \iff z' = t_1/t_2
        \end{equation}
        There is no constraint on $t'$. However, a natural definition would be $t' = 0$, implying $\frac{t'_1 \e}{t'_2 \e} = \frac{t'_1}{t'_2}$. This is only a convenience definition as $w_2$ has no inverse.
        \item $z_2 = 0$ and we divide the (non-zero) non-dual part $z_1$. In this case we have no solution because $z' + t'\e = \frac{z_1}{t_2 \e} \iff z' t_2 \e = z_1$ which is impossible (by hypothesis $z', t_2, z_1 \in \C$).
\end{enumerate}

Division between $w_1$ and $w_2$ is therefore defined for $w_1$ infinitesimal or $w_2$ appreciable.

Exponentiation and rooting of dual-complex are given by

\begin{proposition}
$(z + t\e)^n = z^n + n z^{n-1} t \e$
\end{proposition}
\begin{proof}
$\DC$ is a commutative ring and therefore the binomial theorem holds.

\noindent \begin{equation}
(z + t\e)^n = \sum_{k=0}^n \binom{n}{k} z^{n-k}(t\e)^k = z^n + nz^{n-1}t\e + (t\e)^2 \sum_{k=2}^n \binom{n}{k} z^{n-k}(t\e)^{k-2} = z^n + nz^{n-1}t\e
\end{equation}
\end{proof}

\begin{corollary}
$\sqrt[n]{z + t \e} = \sqrt[n]{z} + (\frac{t}{n(\sqrt[n]{z})^{n-1}}) \e$ for $z \neq 0$.
\end{corollary}

\subsubsection*{Automatic differentiation}

Dual numbers are useful for automatic differentiation. Given an analytic complex-to-complex function $f(z) = \sum_{n=0}^\infty \frac{f^{(n)}(x_0) z^n}{n!}$ on an open set $D$, we extend it by defining $\hat{f}$ on $D + \Z$ as:

\begin{equation}
        \hat{f}(w) \to \sum_{n=0}^\infty \frac{f^{(n)}(0) w^n}{n!}
\end{equation}

\cite{messelmi2015} proves we can always extend a complex analytic to complex-dual numbers like this.
The usefulness of dual numbers for differentiation comes from the following result:

\begin{proposition}[Automatic differentiation]\label{pr:auto}
        Given $f(z)$, a function from complex to complex which is polymorphic on an open set $D$, there exists a unique polymorphic function $\hat{f}(w)$ defined on $D + \Z$ that agrees with $f(z)$ for every $w \in D$. This function $\hat{f}$ is given by
\begin{equation}
        \hat{f}(z + t\e) = f(z) + z f'(z) \e
\end{equation}
for $z \in D$ and $t \in \R$.
\end{proposition}
\begin{proof}
The following proof being simplified and only meant for intuition, see \cite{messelmi2015} for a full formal proof.\ Given $f(z) = \sum_{n=0}^\infty \frac{f^{(n)}(z_0)(z-z_0)^n}{n!} $ and $\hat{f}(z+t\e)$ agreeing with $f$ for complex values and polymorphic on $D+\Z$, we have, by proposition 1,
\begin{equation}
\begin{split}
        \hat{f}(z + t \e) &= \sum_{n=0}^\infty \frac{\hat{f}^{(n)}(z_0) ((z-z_0) + t\e)^n}{n!}
                           = \sum_{n=0}^\infty \frac{f^{(n)}(z_0) ((z-z_0)^n + n(z-z_0)^{n-1}t\e)}{n!} \\
\end{split}
\end{equation}

Which can be rewritten as

\begin{equation}
\begin{split}
        \hat{f}(z + t \e) &= \sum_{n=0}^\infty \frac{f^{(n)}(z_0) (z-z_0)^n}{n!} + \sum_{n=0}^\infty \frac{f^{(n)}(z_0) n(z-z_0)^{n-1}t\e}{n!} \\
                          &= f(z) + t \e (\sum_{n=0}^\infty \frac{f^{(n)}(z_0) (z-z_0)^n}{n!})' \\
                          &= f(z) + t \e f'(z). \\
\end{split}
\end{equation}
\end{proof}

\begin{corollary} Dual-complex exponential, logarithm, sine and cosine are given by

\begin{multicols}{2}
\noindent \begin{equation}\label{eq:expscalar}
 \exp(z + t\e) = \exp(z)(1 + t\e)
\end{equation}

\noindent \begin{equation}
 \log(z + t\e) = \log(z) + \frac t z \e
\end{equation}

\columnbreak
\noindent \begin{equation}
 \sin(z + t\e) = \sin(z) + \cos(z)t\e
\end{equation}

\noindent \begin{equation}
 \cos(z + t\e) = \cos(z) - \sin(z)t\e
\end{equation}
\end{multicols}
\end{corollary}

\section{Linear algebra over dual-complex numbers}

In this section, we recall known and introduce new properties of $\D$ and $\DC$ required to show the consistency of our extension of quantum mechanics' postulates to dual-complex numbers in section 4.

We first introduce in section 3.1 an ordering on dual numbers that will be needed to define a set of valid probabilities. In section 3.2, we motivate a choice of conjugation and norm on $\DC$, establish some properties about them before recalling results about diagonalization of dual-complex Hermitian operators. Finally, we establish that dual-complex unitary operators are also diagonalizable and establish their relation to dual-complex skew-Hermitian operators in the sense that they are the exponential and logarithm of each other.

% Ajouter que ça commence à être des résultats inédits

% \subsubsection*{Ring and quasi-field}
%
% $\DC$ forms a commutative ring with characteristic zero. We also have $\DC/\Z \approx \C$ and $\D/\e\R \approx \R$ which means that $\Z$ and $\e\R$ are maximally ideal.

\subsubsection*{Ordering dual and dual-complex numbers}

We know that while $\R$ can be ordered consistently with the field operations, this is not the case for $\C$. Indeed if $\C$ could be totally ordered we would have either $i \geq 0$ or $-i \geq 0$ and then $(\pm i)^2 = -1 \geq 0$. Orderability of $\D$ depends on the definition of an ordered ring, we choose the usual one: a binary relation $\leq$ such that for all $a, b$ in the ring $a \leq b \implies a + c \leq b + c$ and $0 \leq a, 0 \leq b \implies 0 \leq ab$. \cite{fuchs1963} It turns out that there is a very natural lexicographic ordering for $\D$.

\begin{proposition}
 $\D$ can be totally ordered w.r.\@ to $\R$'s natural ordering and this order is unique up to $\pm \e > 0$.
\end{proposition}
\begin{proof}
Suppose $\leq_\D$ is a total ring order on $\D$ such that for any $a, b \in \R$, $a \leq b \iff a \leq_\D b$. Since $\R[\e] \approx \R[-\e]$ we say wlog that $\e \geq 0$. Now aiming at contradiction, suppose that for some $a \in \R^+$ we have $a \leq \e$. Then, $a^2 \leq \e^2 = 0$ which is a contradiction, implying that for every $a \in \R^+$, $a \geq \e$. This is equivalent to say that $\leq_\D$ is a lexicographic order, i.e.\

\begin{equation}
a_1 + b_1 \e \leq_\D a_2 + b_2 \e \iff a_1 < a_2 \lor (a_1 = a_2 \land b_1 \leq b_2)
\end{equation}

This ordering is total and it is easily seen that it is a ring ordering.
\end{proof}

\subsubsection*{Linear algebra}

\noindent Conjugation for hypercomplex numbers of dimension 2 and imaginary unit $X$ is given by $\bar{a + bX} = a - bX$. For dual-complex, we use the complex conjugation as in \cite{qi2023}. We quickly justify this choice by looking at alternative definitions.

\cite{messelmi2015} suggests four different possible conjugations of interest for us

\begin{multicols}{4}
\noindent \begin{equation}
\bar{w} = \bar{z} + \bar{t}\e
\end{equation}

\columnbreak

\noindent \begin{equation}
\til{w} = z - t\e
\end{equation}

\columnbreak

\noindent \begin{equation}
\bar{\til{w}} = \bar{z} - \bar{t}\e
\end{equation}

\columnbreak

\noindent \begin{equation}
w^* = \bar{z}(1-t\e/z)
\end{equation}
\end{multicols}

$w^*$ is the only possible conjugation that satisfies both $w^*w \in \R$ and $\Re(\Sig(w)) = \Re(\Sig(w^*))$. It is, however, non-linear. Our purpose being to extend quantum mechanics with a unit for infinitesimal variations, this conjugation is unfortunate. The choice of $\bar{w}$ over $\til{w}$ and $\bar{\til{w}}$ is motivated by the idea that we can accept a dual norm as $\e$ represents an infinitesimal variation of it, but not a complex norm.

\begin{definition}
For $n \in \N$, $\DC^n$ represents the module of vectors of length $n$ over $\DC$ equipped with an inner-product:
\begin{equation}
 \braket{\psi|\phi} = \sum_k \bar{\psi_k}\phi_k
\end{equation}

\end{definition}

The norm $|w|$ of $w = z + t\e$ is defined as $|w| = \sqrt{w^* w}$. We can observe the following properties

\begin{multicols}{3}
\noindent \begin{equation}
\bar{w_1 + w_2} = \bar{w_1} + \bar{w_2}
\end{equation}

\noindent \begin{equation}
w + \bar{w} = 2 \Re(z) + 2 \Re(t) \e
\end{equation}

\noindent \begin{equation}
|w| = 0 \iff z = 0
\end{equation}

\columnbreak

\noindent \begin{equation}
\bar{w_1  w_2} = \bar{w_1} ~ \bar{w_2}
\end{equation}

\noindent \begin{equation}
\bar{w}w = |z|^2 + 2 \Re(z\bar{t}) \in \D
\end{equation}

\noindent \begin{equation}\label{eq:geq0}
|w| \geq 0
\end{equation}

\columnbreak

\noindent \begin{equation}
\bar{\bar{w}} = w
\end{equation}

\noindent \begin{equation}\label{eq:modul}
|w| = \sqrt{\bar{w}w} = |z| + \frac{Re(z\bar{t})}{|z|}
\end{equation}

\end{multicols}

In particular equations \ref{eq:geq0} and \ref{eq:modul} entails the following propositions.

\begin{proposition}
 For every $\ket{\psi_\e} = \ket{\psi} + \e \ket{\phi} \in \DC^n$, $\braket{\psi_\e|\psi_\e} = \braket{\psi|\psi} + 2\Re(\braket{\psi|\phi}) \e$ and therefore
\begin{equation}\label{eq:norm}
 ||\ket{\psi_\e}|| = \sqrt{\braket{\psi_\e|\psi_\e}} = ||\ket\psi|| + \frac{\Re(\braket{\psi|\phi})}{||\ket\psi||}\e.
\end{equation}
\end{proposition}
\begin{proposition}
 For any $\ket{\psi} \in \DC^n$, $||\ket\psi|| = \sum_{k=1}^n \psi_k^* \psi_k \geq 0$.
\end{proposition}
%\begin{proof}[]
%\end{proof}

A vector in $\D^n$ or $\DC^n$ is called appreciable when its norm is non-zero or equivalently when at least one of its entries is appreciable. Two vectors are called orthogonal when their inner product is zero and appreciably orthogonal when their inner product is infinitesimal.

We define the adjoint as the transpose conjugate, i.e. $A^\dagger = (A^T)^*$. An operator $H$ is Hermitian if it is self-adjoint, i.e.\ $H^\dagger = H$ and skew-Hermitian if $H^\dagger = -H$. An operator is Hermitian (resp.\ skew-Hermitian) if and only if its significant and infinitesimal are Hermitian (resp.\ skew-Hermitian) and $H$ is Hermitian if and only if $iH$ is skew Hermitian.

An operator $U$ is unitary when $U^\dagger U = I$. The following proposition can be derived from this definition.

\begin{proposition}\label{pr:unitary}
The following propositions are equivalent for $U_\e$ a linear operator:

\begin{enumerate}
 \item[a] $U_\e$ is unitary, i.e.\ $U_\e^\dagger U_\e = I$.
 \item[b] $U_\e U_\e^\dagger = I$.
 \item[c] $U_\e$ preserves the inner product, i.e.\ $(U_\e \ket\psi)^\dagger (U_\e \ket\phi) = \braket{\psi|\phi}$ and therefore the norm $||U_\e \ket\psi|| = ||\ket\psi||$.
 \item[d] Rows of $U_\e$ form an orthonormal basis.
 \item[e] Columns of $U_\e$ form an orthonormal basis.
 \item[f] $U_\e = U - i \e U H$ where $U$ is a complex unitary and $H$ is a complex Hermitian.
\end{enumerate}
\end{proposition}
\begin{proof}
 For point equivalence between point (f) and (a), let $U_\e = A + B\e$.
\begin{equation}
 \begin{split}
  U_\e^\dagger U_\e = I &\iff A^\dagger A + \e A^\dagger B + \e B^\dagger A = I\\
                        &\iff \begin{cases}
      A^\dagger A = I \\
      A^\dagger B + B^\dagger A = 0
   \end{cases}\\
                        &\iff \begin{cases}
      U := A \text{ is a unitary.} \\
      U^\dagger B = - (U^\dagger B)^\dagger
   \end{cases}\\
 \end{split}
\end{equation}

The second condition can be restated as

\begin{equation}
 \begin{split}
      &U^\dagger B = - (U^\dagger B)^\dagger\\
      \iff &H := i U^\dagger B  \text{ is Hermitian.}\\
      \iff &B = -iUH \text{ and } H \text{ is Hermitian.}
 \end{split}
\end{equation}

Points from (a) to (e) are easily seen to be equivalent by following a reasoning analogous to the one for complex unitary operators which conclude our proof.

\end{proof}

In the rest of this section, we prove that the exponential of dual-complex skew-Hermitian are unitary and the logarithms of a dual-complex unitary are skew-Hermitian. We first introduce the exponential of a dual-complex operator, then recall a known result about dual-complex Hermitian diagonalization and finally introduce two new results about dual-complex unitary operators.

\begin{proposition}
The dual-complex matrix exponential always exists.
\end{proposition}
\begin{proof}
For scalar exponentiation in \ref{eq:expscalar}, we used the dual-complex extension from \ref{pr:auto} which works thanks to binomial theorem and, therefore, commutativity. Commutativity is notoriously not a property of linear operators in general. Without commutativity, we get
\begin{equation}
\exp(A + B \e) = \sum_{n=0}^\infty \frac 1 {n!} (A + B\e)^n = \sum_{n=0}^\infty \frac{A^n}{n!} + \e \sum_{n=0}^\infty \frac 1 {n!} \sum_{k=0}^n \binom n k A^k B A^{n-k}
\end{equation}
The significant part is the exponential of the complex matrix $A$ which is known to always converges. The infinitesimal part also converges as
\begin{equation}
\begin{split}
||\sum_{n=0}^\infty \frac 1 {n!} \sum_{k=0}^n \binom n k A^k B A^{n-k}|| &\leq \sum_{n=0}^\infty \frac 1 {n!} \sum_{k=0}^n \binom n k ||A^k||\, ||B||\, ||A^{n-k}||\\
&\leq ||B||\sum_{n=0}^\infty \frac{2^n ||A||^n}{n!}\\
&= ||B|| \exp(2||A||)
\end{split}
\end{equation}
\end{proof}

A proof identical to that of \cite{liu2025} for the case of dual numbers can be used to show that the exponential of dual-complex matrix $A + B\e$ is $\exp(A) + \e L_{\exp}(A, B)$ where $L_{\exp}(A, B)$ is the Fréchet derivative of the exponential function of $A$ in the direction of $B$.

\begin{proposition}[\cite{qi2021}]\label{pr:qi21}
 Suppose $H$ is a Hermitian operator. Then there exists an orthonormal dual-complex basis $\ket{j_\e}$ and $n$ dual eigenvalues $\lambda_j$ of $H$ such that $H = \sum_j \lambda_j \ketbra{j_\e}{j_\e}$.
\end{proposition}

We now establish that dual-complex unitary operators, just like dual-complex Hermitian operators, are diagonalizable and deduce that their logarithms are Hermitian.

\begin{proposition}[Spectral theorem for dual-complex unitary operators]\label{th:specunit}
 A unitary operator is unitarily diagonalizable, i.e.\ for every dual-complex unitary operator $U_\e$, there is a dual-complex orthonormal basis $\ket{j_\e}$ with associated dual-complex eigenvalues $\lambda_j$ such that $U_\e = \sum_j \lambda_j \ketbra{j_\e}{j_\e}$. These eigenvalues are of the form $e^{i\theta_j} + \e e^{i\theta_j} \mu_j$ where $\theta_j$ and $\mu_j$ are real numbers.
\end{proposition}
\begin{proof}
Let $U_\e = U + i\e U H$ be a dual-complex unitary where $U = \exp(iH_0)$ is a complex unitary operator and $H$ is a Hermitian operator. We are going to build a diagonalizing basis for $U_\e$ from a basis of $H + \e H0$.

By proposition \ref{pr:qi21}, we have that $H_0 + \e H$ is diagonalizable. Let $\ket{j_\e} = \ket{j_1} + \e \ket{j_2}$ be an orthonormal basis in which $H_0 + \e H$ is diagonal. Then, $\ket{j_1}$ also forms an orthonormal basis, and we can uniquely write $H = \sum_{jk} h_{jk} \ketbra{j_\e}{k_\e}$ where $h_{jk}^* = -h{kj}$. Since the eigenvalues of $H_0 + \e H$ are of the form $\theta_j + \e \mu_j$ with $\theta_j, \mu_j \in \R$ we have

\begin{equation}
 H_0 \ket{j_2} + H \ket{j_1} = \theta_j \ket{j_2} + \mu_j \ket{j_1} \iff (H_0 - \theta_j I)\ket{j_2} = (\mu_j I - H) \ket{j_1}
\end{equation}

This implies for each $\ket{k_1}$

\begin{equation}
 \bra{k_1}(H_0 - \theta_j I)\ket{j_2} = \bra{k_1}(\mu_j I - H) \ket{j_1} \iff (\theta_k - \theta_j)\braket{k_1|j_2} = -h_{kj}
\end{equation}

Which implies that for each $j \neq k$ such that $\theta_j = \theta_k$, $h_{jk} = 0$.

We will now be able to prove the following claim: we can always construct an orthonormal basis for $U_\e$, whose significant part is $\ket{j_1}$ and with associated eigenvectors whose significant part is $\lambda_j = e^{i\theta_j}$. We need to find vectors $\ket{j_3}$ and complex values $\sigma_j$ such that

\begin{equation}
 (U - \lambda_j I) \ket{j_3} = (\sigma_j I - iUH) \ket{j_1}
\end{equation}

Where $\lambda_j = e^{i\theta_j}$. Since $U = \sum_{k} \lambda_k \ketbra{k_1}{k_1}$, the left side must necessarily be orthogonal to $\ket{j_1}$ which means

\begin{equation}
 \bra{j_1} (\sigma_j I - iUH) \ket{j_1} = 0
 \iff \sigma_j = i \lambda_j h_{jj} \\
\end{equation}

Let $\ket{j_3} = \sum_j \alpha_{jk} \ket{k_1}$, or, equivalently, let $\alpha_{jk} = \braket{k_1|j_3}$. We find for $k \neq j$

\begin{equation}\label{eq:alphj3}
\begin{split}
 \bra{k_1} (U - \lambda_j I) \ket{j_3} = \bra{k_1} (\sigma_j I - iUH) \ket{j_1}
 \iff& (\lambda_k - \lambda_j) \alpha_{jk} = - i \lambda_j h_{kj}
\end{split}
\end{equation}

When $\lambda_j \neq \lambda_k$. When $\lambda_j = \lambda_k$, we showed that $h_{jk} = 0$ and any value of $\alpha_{jk}$ in $\C$ is a solution. Since we always have $\alpha_{jk}$ satisfying \ref{eq:alphj3}, we have a basis. Consider we choose that $\alpha_{jj} = 0$ and $\alpha_{jk} = 0$ when $\lambda_j = \lambda_k$. It happens that the basis $\ket{j_U} := \ket{j_1} + \e \ket{j_3}$ is orthonormal. Indeed

\begin{gather*}
\alpha_{jk}^* = (\lambda_j \lambda_k) \frac{-i\lambda_j^* h_{jk}}{\lambda_k - \lambda_j} = \frac{-i\lambda_k h_{jk}}{\lambda_k - \lambda_j} = -\alpha_{kj}\\
\implies \braket{j_U|k_U} = \braket{j_1|k_1} + \e \braket{j_3|k_1} + \e \braket{j_1|k_3} = \delta_{jk} + \e (\alpha_{kj}^* + \alpha_{jk}) = \delta_{jk}
\end{gather*}

We now showed that $\ket{j_U}$ is an orthonormal basis and by construction we have $\theta_j, \sigma_j \in \R$ and

\begin{equation}
\begin{split}
\sum_j (\lambda_j + \e \sigma_j) \ketbra{j_U}{j_U} &= (\sum_j e^{i\theta_j} \ketbra{j_1}{j_1}) + \e (\sum_j e^{i\theta_j} \ketbra{j_3}{j_3} + \sigma_j \ketbra{j_1}{j_1})\\
                                                   &= U + i \e UH
\end{split}
\end{equation}
\end{proof}

Thanks to the diagonalization of unitary operators, we can establish the following result.

\begin{proposition}\label{pr:hermunit}
The exponential of Hermitian linear operator is a unitary. Moreover, every unitary $U_\e$ is of the form $\exp(iH)$ where $H$ is Hermitian. $H$ is unique modulo $2\pi$.
\end{proposition}
\begin{proof}
To prove the first statement, let $H$ be a Hermitian operator. Then we have

\begin{equation}
 \exp(iH)^\dagger = (\sum_{n=0}^\infty \frac{(iH)^n}{n!})^\dagger = \sum_{n=0}^\infty \frac{(-iH^\dagger)^n}{n!} = \exp(-iH)
\end{equation}

This shows us that the adjoint of the exponential of $iH$ is its inverse:

\begin{equation}
 \exp(iH)^\dagger \exp(iH) = \exp(-iH) \exp(iH) = \exp(-iH+iH) = \exp(0) = I
\end{equation}

For the second statement, let $U_\e = \sum_j (\lambda_j + \e \sigma_j) \ketbra{j_U}{j_U}$ be a dual-complex unitary operator diagonalized as in proposition \ref{th:specunit}. We can write $\lambda_j + \e \sigma_j = e^{i\theta_j}(1 + i\mu_j)$ where $\theta_j, \mu_j \in \R$. Therefore $\log(U_\e) = \sum_j \log(\lambda_j + \e \sigma_j)\ketbra{j_U}{j_U} = \sum_j (i\theta_j + \e i\mu_j)\ketbra{j_U}{j_U}$ which is indeed skew-Hermitian.

\end{proof}

\begin{corollary}
Let $H_1$ and $H_2$ be complex Hermitian matrices, $L_{\exp}(iH_1, iH_2)$ is of the form $iUH$ where $U$ is unitary and $H$ Hermitian.
\end{corollary}

Another result is required to extend the measurement postulate. We say the operator $E$ is appreciably positive if $\braket{\psi|E|\psi}$ positive and appreciable or exactly zero for every vector $\ket\psi$.

\begin{proposition}\label{pr:semipos}
 For every dual-complex linear operator $M_\e$, $M_\e^\dagger M_\e$ is appreciably semipositive.
\end{proposition}
\begin{proof}
Let $\ket\psi$ be an arbitrary vector and $\ket\phi = \sum_j \phi_j \ket{j} = M_\e \ket\psi$ for an orthonormal basis $\ket{j}$. Then $\braket{\psi|M_\e^\dagger M_\e|\psi} = \braket{\phi|\phi} = \sum_j \phi_j^*\phi_j$. From equations 28, 30 and 31 combined, we have that for all $j$, $\phi_j^* \phi_j$ is either appreciable or exactly zero and always greater or equal to zero. It follows that $M_\e^\dagger M_\e$ is appreciably semipositive.
\end{proof}

\section{Postulates of quantum theory over dual-complex numbers}

In this section we introduce an extension of quantum mechnanics postulate to allow for dual-complex amplitudes and operators.
We replace the usual notion of state by that of an ``$\e$-state'', a dual-complex unit vector in the dual-complex extension of a Hilbert-space. We extend the usual notions of unitary evolution and Hermitian observables by allowing dual-complex unitary ``$\e$-evolutions'' and dual-complex Hermitian ``$\e$-observables''.
These notions will allow us to study linear approximation of quantum states, which we will call ``$\e$-quantum states.''.

We present a suggestion of extension of the usual postulates of quantum theory, then justify them quickly.

\paragraph{Postulate 1.} An $\e$-state vector is a unit dual-complex vector. At every time, a dual-complex quantum system is described by an $\e$-state.
\paragraph{Postulate 2.} An operator $U$ represents a valid $\e$-evolution if and only if $U$ is unitary. The evolution of a closed $\e$-quantum system is described by an $\e$-evolution.
\paragraph{Postulate 3.} A set of operators $\{M_m\}$ is an $\e$-measurement operator collection if it is bound by the completeness relation. The evolution of an open $\e$-quantum system is described by such a collection. The outcome $m$ appears with the dual probability $p(m) = \bra\psi M_m^\dagger M_m \ket\psi$. The resulting $\e$-state is

\begin{equation}\label{def:meas}
 \ket{\psi'} = \frac{M_m \ket\psi}{\sqrt{p(m)}}
\end{equation}

\paragraph{Postulate 4.} Given two $\e$-states representing two systems, the composite system is represented by the tensor product of these two $\e$-states.
\\

\subsubsection*{State space}

Postulate 1 aims at capturing the notion of the extension of a state vector to the dual-complex domain. The goal being to take a linear approximation of a system according to a certain parameter, we start at time zero with a usual complex unit vector that we evolve according to postulates 2 and 3, and so our states will always keep a norm equal to $1$.

If we choose to start from a dual-complex vector, since the infinitesimal part of the vector represents the derivative of the significant part according to some parameter, equation \ref{eq:norm} tells us that the norm is one if and only if the derivation operator is skew-Hermitian.
% \begin{equation}
% \begin{split}
%  \braket{\psi(t+\e z)|\psi(t+\e z)} &= \braket{\psi(t)|\psi(t)} + 2 z \Re(\braket{\psi(t)|\dstate \psi t}) \e\\
%                                     &= 1 + 2 z \Re(\braket{\psi(t)|(-iH)|\psi(t)}) \e\\
%                                     &= 1.\\
% \end{split}
% \end{equation}
%
% A second concern is when we extend space. Extending an analytic wavefunction to take a dual input can in theory lead to a norm that is not 1. However, when our wave function tends to zero when the position goes to $\pm \infty$, which is the case in an actual physical system, the derivative in space will always be skew-Hermitian. Say, if at a given time, we have the vector $\ket{\Psi} = \int_{-\infty}^\infty \Psi(x)\ket{x}\mathrm{d}x$ with boundary conditions
%
% \begin{equation}
%  \lim_{x \to -\infty} \Psi(x) = \lim_{x \to +\infty} \Psi(x) = 0
% \end{equation}
%
% Then
%
% \begin{equation}
% \begin{split}
%  \braket{\Psi| \frac{\partial}{\partial x} | \Psi} &= \int_{-\infty}^\infty \Psi(x)^* \frac{\partial \Psi(x)}{\partial x} \mathrm{d}x\\
%  &= \left. \braket{\Psi(x)|\Psi(x)} \right|_{-\infty}^\infty - \int_{-\infty}^\infty (\frac{\partial \Psi(x)}{\partial x})^* \Psi(x) \mathrm{d}x\\
%  &= - (\braket{\Psi | \frac{\partial}{\partial x} | \Psi})^*
% \end{split}
% \end{equation}
%
% Therefore the derivative operator is skew-hermitian and we always have $\braket{\Psi | \frac{\partial \Psi}{\partial x}} \in i\R$. This means that whenever the system is physical, it can be extended while keeping norm 1 at all time. We can therefore see our first extended postulate as ``physical''.

\subsubsection*{Evolution}

This postulate is motivated by the fact that given an analytic path of unitary $U(z)$, its dual-complex extension will be unitary (which is easily seen from the fact that the derivative $U(z)$ is of the form $iU(z)H(z)$ where $H(z)$ is Hermitian). $\e$-evolutions represent therefore a first-order approximation of traditional unitary evolutions.

In traditional quantum mechanics, this postulate is equivalently expressed through the Schrödinger equation. From proposition \ref{pr:hermunit}, we can restate our extension of postulate 2 as

\paragraph{Postulate 2'.} The evolution of a closed system is described by the Schrödinger equation,
\begin{equation}
 i\hbar \dstate \psi t = H\ket\psi
\end{equation}

Where $\ket\psi$ is an $\e$-state, $H$ a dual-complex Hermitian linear operator and $\hbar$ is the Planck's constant.

\subsubsection*{Measurement}

Similar to postulate 2, we extend general measurements with dual-complex operators by taking a collection of analytic measurement operators and extend them into $\DC$. It can easily be seen that since $\sum_m M_m^\dagger M_m = I$, their derivatives $\dot{M_m}$ is such that $\sum_m M_m^\dagger \dot{M_m}$ is skew-Hermitian. It can also be seen that for an $\e$-quantum measurement operators collection $\{M_m + \e N_m\}$, the completeness relation $\sum_m (M_m + \e N_m)^\dagger (M_m + \e N_m)$ can be restated as

\begin{multicols}{2}
\noindent \begin{equation}
 \text{(i) } \sum_m M_m^\dagger M_m = I
\end{equation}
\columnbreak
\noindent \begin{equation}
 \text{(ii) } \sum_m M_m^\dagger N_m \text{ is skew-Hermitian.}
\end{equation}
\end{multicols}

The traditional postulate for measurements also defines what are the resulting states and with which probability. To ensure our extension doesn't create unwanted behaviors, we must check that probabilities are positive and sum to one. We must also check that the resulting state is well defined as it is renormalized by a probability that could maybe be infinitesimal. Fortunately, such behaviors do not arise, as we show in the next section (\ref{pr:consistency}).

In particular, as in traditional quantum theory, we can see that when we have exactly one measurement operator, it is a unitary, which means that the extended postulate 2 can be reduced to the postulate 3.

\subsubsection*{Composite systems}

Composing systems is done in the traditional way, using tensor products. Again the postulate preserves the unit norm. If we have two $\e$-states $\ket\psi$ and $\ket\phi$, their norm will be $\sqrt{\braket\psi\psi \braket\phi\phi} = 1$.

\subsubsection*{Density operators formalism}

We use the usual notion of density operators, which is easily seen to be compatible with the extended postulates. An $\e$-density operator $\rho$ is a dual-complex, unit-trace, positive operator. As in the traditional case, applying a unitary $U$ to $\rho$ yields $U \rho U^\dagger$ and measuring $\rho$ with $\{M_m\}$ yields $\sum_m M_m \rho M_m^\dagger$.

\section{Consistency of the extended postulates}

In this section, we show that the evolution of a system preserves the norm of the state vector (self-consistency) and that our postulates provide a first-order approximation of quantum systems.

\subsubsection*{Self-consistency}
\todo[inline]{
Mettre un théorème qui dit que les $\e$-states sont stables sous les opérations physiques admises par $\e$-evolution et $\e$-measurement. Probabilités somment à 1 et sont positives. }

\begin{proposition}\label{pr:consistency}
A system that starts from an $\e$-states and is evolved by $\e$-evolutions and $\e$-measurements is described by an $\e$-state at all time, i.e.\ if there's a sequence $\ket{\psi_0}, \ket{\psi_1}, \dots, \ket{\psi_k}$ such that $\ket{\psi_0}$ is an $\e$-state and $\ket{\psi_{j+1}}$ was obtained either by applying an $\e$-unitary or an $\e$-measurement to $\ket{\psi_j}$, every $\ket{\psi_j}$ is an $\e$-state. Moreover, probabilities from any $\e$-measurement are either zero or appreciably positive and sum to one.
\end{proposition}
\begin{proof}
By proposition \ref{pr:unitary}, a dual-complex unitary preserves the norm. For measurements, from \ref{def:meas} norm preservation is given by
\begin{equation}
\norm{\frac{M_m \ket\psi}{\sqrt{p(m)}}} = \norm{\frac{M_m \ket\psi}{\sqrt{\braket{\psi|M_m^\dagger M_m|\psi}}}} = \frac{||M_m \ket\psi||}{||M_m\psi||} = 1.
\end{equation}
Probabilities sum to one since $\sum_m p(m) = \sum_m \braket{\psi|M_m^\dagger M_m|\psi} = \braket{\psi|(\sum_m M_m^\dagger M_m)|\psi} = \braket{\psi|I|\psi} = 1$ and are either zero or appreciably positive by proposition \ref{pr:semipos}.
\end{proof}

\subsubsection*{Consistency with traditional quantum theory}

\todo[inline]{paramétrer les unitaires avec des parenthèses}

\todo[inline]{Écrire la réciproque + rajouter des définitions pour tout faire dans la preuve de la proposition}

\begin{definition}
We introduce the dual-complex extension of a quantum operation and the correction of dual-complex quantum operation.
\begin{itemize}
 \item Let $\E_z(\rho)$ be a quantum operation analytic in $z$ whose operator-sum representation is, for some basis $\ket{e_k}$ and unitary operator $U_z$,
\begin{equation}\label{eq:opsum}
\E_z(\rho) = tr_B(U_z [\rho \otimes \ketbra{e_0}{e_0}] U_z^\dagger)
\end{equation}

Let $\dot{\E_z} = \frac{\mathrm{d}\E_z}{\mathrm{d}z}$. We call $\E$'s dual-complex extension the dual-complex operation

\begin{equation}
\F_{z+\dot{z}\e}(\rho) = \E_z(\rho) + \dot{z}\e \dot{\E}_z(\rho)
\end{equation}

 \item Let $\F_w(\rho)$ be a dual-complex extension of a quantum operation $\E_z(\rho)$ as in \ref{eq:opsum} where $w = z + \dot{z}\e$. Then $\F_w$ has the form
        \begin{equation}
        \F_{w}(\rho) = tr_B(V_w [\rho \otimes \ketbra{e_0}{e_0}] V_w^\dagger)
        \end{equation}
  where $V_w = U_z + \e \dot{z} \dot{U_z}$, $\dot{U_z} = \frac{\mathrm{d}U_z}{\mathrm{d}z} = iU_zH$ and $H$ is Hermitian. $\til{V}_w(h)$ is the complex correction of $V_w$ with parameter $h \in \C$, defined as
        \begin{equation}
        \til{V}_w(h) = U_z + i \dot{z} h {U_z} H + h^2\sum_{n=0}^\infty \frac{1}{n!} (iH)^{n+2}h^n
        \end{equation}
  $\til{\F}_w(\rho, h)$ is the complex correction of $\F_w(\rho, h)$ with parameter $h \in \C$, defined as
        \begin{equation}\label{eq:corrF}
        \til{F}_w(\rho, h) = tr_B(\til{V}_w(h) [\rho \otimes \ketbra{e_0}{e_0}] \til{V}_w(h)^\dagger)
        \end{equation}
\end{itemize}

\end{definition}

\begin{proposition}
Let $\E_z(\rho)$ and $\F_w(\rho)$ be a quantum operation and its dual-complex extension. For any parameter $h \in \C$, the correction $\til{F}_w(\rho, h)$ is a valid quantum operation and $\til{F}_{z + \dot{z}\e}(\rho, h) = \E_z(\rho) + O(h^2)$.
\end{proposition}

\begin{proof}
First, observe that $\til{V}_w(h)$ is a unitary
\begin{equation}
\til{V}_w(h) = U_z + i \dot{z} h U_z H + h^2 \sum_{n=0}^\infty \frac{1}{n!} (iH)^{n+2}h^n = U_z(\sum_{n=0}^\infty \frac{1}{n!} (iHh)^{n}) = U_z \exp(iHh).
\end{equation}

It immediately follows that \ref{eq:corrF} is a quantum operation. For the second claim, we can see that

\begin{equation}
\begin{split}
\end{split}
\begin{align*}
\mathcal{E}_{z+\dot{z}h}(\rho)
&= \mathcal{E}_z(\rho) + \dot{z} h \dot{\mathcal{E}}(\rho) + O(h^2) \\
\tilde{F}_{z+\dot{z}\e}(\rho, h)
&= \operatorname{tr}_B\left(\til{V}_{z+\dot{z}\e}(h) \left[ \rho \otimes \ketbra{e_0}{e_0} \right] \til{V}_{z+\dot{z}\e}(h)^\dagger\right) \\
&= \operatorname{tr}_B\left(\left( U_z + \dot{z} h \dot{U}_z + O(h^2) \right)
\left[ \rho \otimes \ketbra{e_0}{e_0} \right]
\left( U_z + \dot{z} h \dot{U}_z + O(h^2) \right)^\dagger\right)\\
&=\operatorname{tr}_B\left(U_z \left[ \rho \otimes \ketbra{e_0}{e_0} \right] U_z^\dagger\right) \\
&+ \operatorname{tr}_B\left(U_z \left[ \rho \otimes \ketbra{e_0}{e_0} \right] (\dot{z} h \dot{U}_z)^\dagger
+ (\dot{z} h \dot{U}_z) \left[ \rho \otimes \ketbra{e_0}{e_0} \right] U_z^\dagger \right) + O(h^2)\\
&= \mathcal{E}_z(\rho) + \dot{z} h \dot{\mathcal{E}}_z(\rho) + O(h^2)
\end{align*}
\end{equation}
\end{proof}

\section{Application example: Quantum Cellular Automata}

\todo[inline]{Rappeler le contexte}
\todo[inline]{Ajouter la figure}

In this section, we introduce an example of analytic quantum systems from \cite{arrighi2020}, present their dual-complex extension and go back to traditional quantum systems.

The Dirac QCA is

The Quantum Cellular Automata is defined as in fig 1. All gates are identical and defined by

\begin{equation}
W_h = \begin{pmatrix}
1 & 0 & 0 & 0\\
0 & -is & c & 0\\
0 & c & -is & 0\\
0 & 0 & 0 & 1\\
\end{pmatrix}
\end{equation}

where $s = \sin(mh)$ and $c = \cos(mh)$. $W_h$ is approximated by

\begin{equation}
W_\e = \begin{pmatrix}
1 & 0 & 0 & 0\\
0 & -im\e & 1 & 0\\
0 & 1 & -im\e & 0\\
0 & 0 & 0 & 1\\
\end{pmatrix}
\end{equation}

The associated gate is

\begin{equation}
W'_h = e^h \begin{pmatrix}
1 & 0 & 0 & 0\\
0 & 0 & e^{-im} & 0\\
0 & e^{-im} & 0 & 0\\
0 & 0 & 0 & 1\\
\end{pmatrix}
\end{equation}

\todo[inline]{Montrer la limite, faire la limite vers l'équation de Dirac grâce à l'AD}

\todo[inline]{Montrer que c'est stricement Lorentz covariant au sens de Lorentz discrete covariance du papier + faire le lien entre manière discrète de montrer que l'équation continue est covariante (au sens continu).}


% \section{Alternative postulates}
%
% \subsubsection*{Pathological wave function}
%
% It is possible to build pathological normalizable, analyticl and continuous wave functions such that $r := \left. \braket{\Psi(x)|\Psi(x)} \right|_{-\infty}^\infty \neq 0$. In this case, we get $||\Psi(x + y \e)|| = 1 + y r \e$.
%
% \paragraph{Relaxed Postulate 1.} A pathological $\e$-state is a vector $\ket\psi$ in $\DC^n$ such that $||\ket\psi|| = 1 \mod \Z$.
%
% To preserve pathological $\e$-stateness, a valid operator $U_\e = U + \e V$ need only to satisfy $U$ unitary, as $U_\e = U \mod \Z$.
%
% \paragraph{Relaxed Postulate 2.} An $\e$-evolution of a pathological system is given by $U_\e = U + \e V$ where $U$ is unitary and $V$ is any linear operator.
%
% For measurement, probabilities should once again sum to one. But it would mean $U_\e^\dagger U_\e = I$, which sends us back to the well-behaved case. An extension modulo $\Z$ would be
%
% \paragraph{Relaxed Postulate 3.} A measurement is defined by a collection $\{M_m\}$ of measurement operators with $\sum_m M_m^\dagger M_m = I \mod \Z$.
%
% With the consequence of probabilities summing to $1 + \Z$, leading to negative infinitesimal probabilities (if $P(A) = 1 + \e$, $P(\bar{A}) = 1 - 1 - \e = -\e$).
%
% For postulate 4, we satisfy ourselves with noting that $||\ket\psi|| = ||\ket\phi|| = 1 \mod O_\e$ implies $||\ket\psi \otimes \ket\phi|| = 1 \mod O_\e$.

\bibliographystyle{johd}
\bibliography{bib}

\end{document}
