\batchmode


\documentclass{article}
\RequirePackage{ifthen}


\usepackage[english]{babel}
\usepackage[utf8]{inputenc}
\usepackage{amssymb}
\usepackage{amsmath}
\usepackage{amsthm}
\usepackage{johd}
\usepackage{braket}
\usepackage{multicol}


\newtheorem{theorem}{Theorem} 
\newtheorem{lemma}[theorem]{Lemma} 
\newtheorem{definition}[theorem]{Definition} 
\newtheorem{proposition}[theorem]{Proposition} 
\newtheorem{corollary}[theorem]{Corollary} 
%
\providecommand{\N}{\mathbb{N}}%
\providecommand{\R}{\mathbb{R}}%
\providecommand{\C}{\mathbb{C}}%
\providecommand{\D}{\mathbb{D}}%
\providecommand{\DC}{\mathbb{DC}}%
\providecommand{\GD}{\mathbb{GD}}%
\providecommand{\GDC}{\mathbb{GDC}}%
\providecommand{\Z}{\mathcal{O}_\epsilon }%
\providecommand{\e}{\epsilon} 

%
\providecommand{\E}{\mathcal{E}}%
\providecommand{\til}{\widetilde} 
%
\renewcommand{\bar}{\overline}

%
\renewcommand{\Re}{\operatorname{Re}}
%
\renewcommand{\Im}{\operatorname{Im}}%
\providecommand{\Sig}{\operatorname{Sig}}%
\providecommand{\Inf}{\operatorname{Inf}} 

\title{Postulates of quantum mechanics with dual-complex numbers}

\author{Nathan Houyet$^{a}$, Dogukan Bakircioglu$^{b}$, Pablo Arrighi$^{b}$\  \\
        \small $^{a}$Université Paris-Saclay (student), France \\
        \small $^{b}$QuaCS, LMF, France \\
}
\date{}



\usepackage{xcolor}



\makeatletter

\makeatletter
\count@=\the\catcode`\_ \catcode`\_=8 
\newenvironment{tex2html_wrap}{}{}%
\catcode`\<=12\catcode`\_=\count@
\newcommand{\providedcommand}[1]{\expandafter\providecommand\csname #1\endcsname}%
\newcommand{\renewedcommand}[1]{\expandafter\providecommand\csname #1\endcsname{}%
  \expandafter\renewcommand\csname #1\endcsname}%
\newcommand{\newedenvironment}[1]{\newenvironment{#1}{}{}\renewenvironment{#1}}%
\let\newedcommand\renewedcommand
\let\renewedenvironment\newedenvironment
\makeatother
\let\mathon=$
\let\mathoff=$
\ifx\AtBeginDocument\undefined \newcommand{\AtBeginDocument}[1]{}\fi
\newbox\sizebox
\setlength{\hoffset}{0pt}\setlength{\voffset}{0pt}
\addtolength{\textheight}{\footskip}\setlength{\footskip}{0pt}
\addtolength{\textheight}{\topmargin}\setlength{\topmargin}{0pt}
\addtolength{\textheight}{\headheight}\setlength{\headheight}{0pt}
\addtolength{\textheight}{\headsep}\setlength{\headsep}{0pt}
\newwrite\lthtmlwrite
\makeatletter
\let\realnormalsize=\normalsize
\global\topskip=2sp
\def\preveqno{}\let\real@float=\@float \let\realend@float=\end@float
\def\@float{\let\@savefreelist\@freelist\real@float}
\def\liih@math{\ifmmode$\else\bad@math\fi}
\def\end@float{\realend@float\global\let\@freelist\@savefreelist}
\let\real@dbflt=\@dbflt \let\end@dblfloat=\end@float
\let\@largefloatcheck=\relax
\let\if@boxedmulticols=\iftrue
\def\@dbflt{\let\@savefreelist\@freelist\real@dbflt}
\def\adjustnormalsize{\def\normalsize{\mathsurround=0pt \realnormalsize
 \parindent=0pt\abovedisplayskip=0pt\belowdisplayskip=0pt}%
 \def\phantompar{\csname par\endcsname}\normalsize}%
\def\lthtmltypeout#1{{\let\protect\string \immediate\write\lthtmlwrite{#1}}}%
\usepackage[tightpage,active]{preview}
\PreviewBorder=0.5bp
\newbox\lthtmlPageBox
\newdimen\lthtmlCropMarkHeight
\newdimen\lthtmlCropMarkDepth
\long\def\lthtmlTightVBoxA#1{\def\lthtmllabel{#1}
    \setbox\lthtmlPageBox\vbox\bgroup\catcode`\_=8 }%
\long\def\lthtmlTightVBoxZ{\egroup
    \lthtmlCropMarkHeight=\ht\lthtmlPageBox \advance \lthtmlCropMarkHeight 6pt
    \lthtmlCropMarkDepth=\dp\lthtmlPageBox
    \lthtmltypeout{^^J:\lthtmllabel:lthtmlCropMarkHeight:=\the\lthtmlCropMarkHeight}%
    \lthtmltypeout{^^J:\lthtmllabel:lthtmlCropMarkDepth:=\the\lthtmlCropMarkDepth:1ex:=\the \dimexpr 1ex}%
    \begin{preview}\copy\lthtmlPageBox\end{preview}}%
\long\def\lthtmlTightFBoxA#1{\def\lthtmllabel{#1}%
    \adjustnormalsize\setbox\lthtmlPageBox=\vbox\bgroup\hbox\bgroup %
    \let\ifinner=\iffalse \let\)\liih@math %
    \bgroup\catcode`\_=8 }%
\long\def\lthtmlTightFBoxZ{\egroup\egroup
    \@next\next\@currlist{}{\def\next{\voidb@x}}%
    \expandafter\box\next\egroup %
    \lthtmlCropMarkHeight=\ht\lthtmlPageBox \advance \lthtmlCropMarkHeight 6pt
    \lthtmlCropMarkDepth=\dp\lthtmlPageBox
    \lthtmltypeout{^^J:\lthtmllabel:lthtmlCropMarkHeight:=\the\lthtmlCropMarkHeight}%
    \lthtmltypeout{^^J:\lthtmllabel:lthtmlCropMarkDepth:=\the\lthtmlCropMarkDepth:1ex:=\the \dimexpr 1ex}%
    \begin{preview}\copy\lthtmlPageBox\end{preview}}%
    \long\def\lthtmlinlinemathA#1#2\lthtmlindisplaymathZ{\lthtmlTightVBoxA{#1}{\hbox\bgroup#2\egroup}\lthtmlTightVBoxZ}
    \def\lthtmlinlineA#1#2\lthtmlinlineZ{\lthtmlTightVBoxA{#1}{\hbox\bgroup#2\egroup}\lthtmlTightVBoxZ}
    \long\def\lthtmldisplayA#1#2\lthtmldisplayZ{\lthtmlTightVBoxA{#1}{#2}\lthtmlTightVBoxZ}
    \long\def\lthtmldisplayB#1#2\lthtmldisplayZ{\\edef\preveqno{(\theequation)}%
        \lthtmlTightVBoxA{#1}{\let\@eqnnum\relax#2}\lthtmlTightVBoxZ}
    \long\def\lthtmlfigureA#1{\let\@savefreelist\@freelist
        \lthtmlTightFBoxA{#1}}
    \long\def\lthtmlfigureZ{
        \lthtmlTightFBoxZ\global\let\@freelist\@savefreelist}
    \long\def\lthtmlpictureA#1{\let\@savefreelist\@freelist
        \lthtmlTightVBoxA{#1}}
    \long\def\lthtmlpictureZ{
        \lthtmlTightVBoxZ\global\let\@freelist\@savefreelist}
\def\lthtmlcheckvsize{\ifdim\ht\sizebox<\vsize 
  \ifdim\wd\sizebox<\hsize\expandafter\hfill\fi \expandafter\vfill
  \else\expandafter\vss\fi}%
\providecommand{\selectlanguage}[1]{}%
\makeatletter \tracingstats = 1 
\providecommand{\Alpha}{\textrm{A}}
\providecommand{\Beta}{\textrm{B}}
\providecommand{\Chi}{\textrm{X}}
\providecommand{\Epsilon}{\textrm{E}}
\providecommand{\Eta}{\textrm{H}}
\providecommand{\Iota}{\textrm{J}}
\providecommand{\Kappa}{\textrm{K}}
\providecommand{\Mu}{\textrm{M}}
\providecommand{\Nu}{\textrm{N}}
\providecommand{\Omicron}{\textrm{O}}
\providecommand{\Rho}{\textrm{R}}
\providecommand{\Tau}{\textrm{T}}
\providecommand{\Zeta}{\textrm{Z}}
\providecommand{\omicron}{\textrm{o}}


\begin{document}
\pagestyle{empty}\thispagestyle{empty}\lthtmltypeout{}%
\lthtmltypeout{latex2htmlLength hsize=\the\hsize}\lthtmltypeout{}%
\lthtmltypeout{latex2htmlLength vsize=\the\vsize}\lthtmltypeout{}%
\lthtmltypeout{latex2htmlLength hoffset=\the\hoffset}\lthtmltypeout{}%
\lthtmltypeout{latex2htmlLength voffset=\the\voffset}\lthtmltypeout{}%
\lthtmltypeout{latex2htmlLength topmargin=\the\topmargin}\lthtmltypeout{}%
\lthtmltypeout{latex2htmlLength topskip=\the\topskip}\lthtmltypeout{}%
\lthtmltypeout{latex2htmlLength headheight=\the\headheight}\lthtmltypeout{}%
\lthtmltypeout{latex2htmlLength headsep=\the\headsep}\lthtmltypeout{}%
\lthtmltypeout{latex2htmlLength parskip=\the\parskip}\lthtmltypeout{}%
\lthtmltypeout{latex2htmlLength oddsidemargin=\the\oddsidemargin}\lthtmltypeout{}%
\makeatletter
\if@twoside\lthtmltypeout{latex2htmlLength evensidemargin=\the\evensidemargin}%
\else\lthtmltypeout{latex2htmlLength evensidemargin=\the\oddsidemargin}\fi%
\lthtmltypeout{}%
\makeatother
\setcounter{page}{1}
\onecolumn

% !!! IMAGES START HERE !!!

\stepcounter{section}
\stepcounter{subsection}
\stepcounter{subsection}
{\newpage\clearpage
\lthtmlinlinemathA{tex2html_wrap_indisplay1730}%
$\displaystyle \frac{z_1 + t_1 \epsilon }{z_2 + t_2 \epsilon } = \frac{(z_1 + t_1 \epsilon )(z_2 - t_2 \epsilon )}{(z_2 + t_2 \epsilon )(z_2 - t_2 \epsilon )} = \frac{z_1 z_2 - z_1 t_2 \epsilon + t_1 z_2 \epsilon }{z_2^2} = \frac{z_1}{z_2} + \frac{z_2 t_1 - z_1 t_2 }{z_2^2} \epsilon$%
\lthtmlindisplaymathZ
\lthtmlcheckvsize\clearpage}

\stepcounter{section}
\stepcounter{subsection}
{\newpage\clearpage
\lthtmlinlinemathA{tex2html_wrap_indisplay1782}%
$\displaystyle (z + t\epsilon )^n = \sum_{k=0}^n \binom{N}{k} z^{n-k}(t\epsilon )^k = z^n + nz^{n-1}t\epsilon + (t\epsilon )^2 \sum_{k=2}^n \binom{N}{k} z^{n-k}(t\epsilon )^{k-2} = z^n + nz^{n-1}t\epsilon$%
\lthtmlindisplaymathZ
\lthtmlcheckvsize\clearpage}

\stepcounter{subsection}
\stepcounter{subsection}
{\newpage\clearpage
\lthtmlinlinemathA{indisplay1835}%
\begin{indisplay}\begin{split}
\hat{f}(z + t \epsilon ) &= \sum_{n=0}^\infty \frac{f^{(n)}(0) (z + t\epsilon )^n}{n!} \\
&= \sum_{n=0}^\infty \frac{f^{(n)}(0) (z^n + nz^{n-1}t\epsilon )}{n!} \\
\end{split}\end{indisplay}%
\lthtmlindisplaymathZ
\lthtmlcheckvsize\clearpage}

{\newpage\clearpage
\lthtmlinlinemathA{indisplay1837}%
\begin{indisplay}\begin{split}
\hat{f}(z + t \epsilon ) &= \sum_{n=0}^\infty \frac{f^{(n)}(0) z^n}{n!} + \sum_{n=0}^\infty \frac{f^{(n)}(0) nz^{n-1}t\epsilon }{n!} \\
&= f(z) + t \epsilon (\sum_{n=0}^\infty \frac{f^{(n)}(0) z^n}{n!})' \\
&= f(z) + t \epsilon f'(z). \\
\end{split}\end{indisplay}%
\lthtmlindisplaymathZ
\lthtmlcheckvsize\clearpage}

\stepcounter{section}
\stepcounter{subsection}
\stepcounter{subsection}
\stepcounter{subsection}
\stepcounter{section}
\stepcounter{subsection}
\stepcounter{paragraph}
{\newpage\clearpage
\lthtmlinlinemathA{indisplay2004}%
\begin{indisplay}\begin{split}
|| \ket{\psi(t + y \epsilon )} || = 1 &\iff \braket{\psi(t + y \epsilon ) | \psi(t + y \epsilon )} = 1 \\
&\iff \braket{\psi(t) | \psi(t)} + y \epsilon \braket{\psi'(t) | \psi(t)} + y \epsilon \braket{\psi(t) | \psi'(t)} + y^2 \epsilon ^2 \braket{\psi'(t) | \psi'(t)} = 1 \\
&\iff \braket{\psi(t) | \psi(t)} + 2 y \epsilon Re(\braket{\psi'(t) | \psi(t)}) = 1\\
&\iff \begin{cases}
\text{(i) } \braket{\psi(t) | \psi(t)} \\
\text{(ii) } \braket{\psi'(t) | \psi(t)} \in i\mathbb{R}\end{cases}
\end{split}\end{indisplay}%
\lthtmlindisplaymathZ
\lthtmlcheckvsize\clearpage}

{\newpage\clearpage
\lthtmlinlinemathA{indisplay2014}%
\begin{indisplay}\begin{split}
\braket{\Psi(t) | \frac{\partial}{\partial x} | \Psi(t)} &= \int_{-\infty}^\infty \Psi(x, t)^* \frac{\partial \Psi(x, t)}{\partial x} \mathrm{d}x\\
&= \left. \braket{\Psi(x, t)|\Psi(x, t)} \right|_{-\infty}^\infty - \int_{-\infty}^\infty (\frac{\partial \Psi(x, t)}{\partial x})^* \Psi(x, t) \mathrm{d}x\\
&= - (\braket{\Psi(t) | \frac{\partial}{\partial x} | \Psi(t)})^*
\end{split}\end{indisplay}%
\lthtmlindisplaymathZ
\lthtmlcheckvsize\clearpage}

\stepcounter{subsection}
{\newpage\clearpage
\lthtmlinlinemathA{indisplay2039}%
\begin{indisplay}\begin{split}
U_\epsilon ^\dagger U_\epsilon = I &\iff A^\dagger A + \epsilon A^\dagger B + \epsilon B^\dagger A = I\\
&\iff \begin{cases}
\text{(i) } A^\dagger A = I \\
\text{(ii) } A^\dagger B + B^\dagger A = 0
\end{cases}\\
&\iff \begin{cases}
\text{(i) } U := A \text{ is a unitary.} \\
\text{(ii) } U^\dagger B = - (U^\dagger B)^\dagger
\end{cases}\\
\end{split}\end{indisplay}%
\lthtmlindisplaymathZ
\lthtmlcheckvsize\clearpage}

\stepcounter{paragraph}
\stepcounter{subsection}
\stepcounter{paragraph}
\stepcounter{subsection}
\stepcounter{paragraph}
\stepcounter{section}
\stepcounter{subsection}
\stepcounter{paragraph}
\stepcounter{paragraph}
\stepcounter{paragraph}
\stepcounter{subsection}

\end{document}
