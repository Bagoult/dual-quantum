%% Short data paper template
%% Created by Simon Hengchen and Nilo Pedrazzini for the Journal of Open Humanities Data (https://openhumanitiesdata.metajnl.com)

\documentclass{article}
\usepackage[english]{babel}
\usepackage[utf8]{inputenc}
\usepackage{amssymb}
\usepackage{amsmath}
\usepackage{johd}
\usepackage{braket}

\newcommand{\R}{\mathbb{R}}
\newcommand{\C}{\mathbb{C}}
\newcommand{\D}{\mathbb{D}}
\newcommand{\DC}{\mathbb{DC}}
\newcommand{\A}{\mathcal{A}}
\newcommand{\e}{\epsilon}
\newcommand{\til}{\widetilde}
\renewcommand{\bar}{\overline}
\newcommand{\dd}[1]{\mathrm{d}#1}

\title{Postulates of quantum mechanics with dual-complex numbers}

\author{Nathan Houyet$^{a}$, Dogukan Bakircioglu$^{b}$, Pablo Arrighi$^{b}$ \\
        \small $^{a}$Université Paris-Saclay (student), France \\
        \small $^{b}$QuaCS, LMF, France \\
}
\date{}

\begin{document}
\maketitle

\section{Introduction}
\noindent Adding an imaginary unit $i$ with $i^2 = -1$ to real numbers gives rise to the complex numbers. Other definition of this imaginary unit can lead to useful numbers systems, like the dual numbers with $\e^2 = 0$. Complex-dual numbers are defined by adding both imaginary units to reals. We have:

\begin{equation}
\C = \R [i]/\langle i^2+1 \rangle
\end{equation}
\begin{equation}
\D = \R [\e]/\langle \e^2 \rangle
\end{equation}
\begin{equation}
\DC = \C [\e]/\langle \e^2 \rangle
\end{equation}

We usually denote complex numbers by the form $z = a + bi$ and complex-dual numbers by the form $w = z + t\e$. We use the terms "imaginary-complex" and "imaginary-dual" to distinguish between numbers of the form $bi$ and $t\e$. We introduce the following notation

\begin{equation}
Du: \DC \to \C: w = z + t\e \to (w \e)/\e = t
\end{equation}

\begin{equation}
ND: \DC \to \C: w = z + t\e \to w - Du(w) \e = z
\end{equation}

$Im(w)$ is ambiguous because their are two imaginary units in complex-duals. We use the following notation. For a number $w = z + t\e$ with $z = a + bi$ and $t = c + di$ we write

\begin{equation}
R: \DC \to \R: w \to Re(ND(w)) = a
\end{equation}

\begin{equation}
C: \DC \to \R: w \to Im(ND(w)) = b
\end{equation}

\begin{equation}
D: \DC \to \R: w \to Re(Du(w)) = c
\end{equation}

\begin{equation}
DC: \DC \to \R: w \to Im(Du(w)) = d
\end{equation}

\subsection{Operations on dual-complex numbers}

Dual-complex numbers form a commutative ring with characteristic zero, but not a field. We define addition and division

\begin{equation}
(z_1 + t_1 \e) + (z_2 + t_2 \e) = (z_1 + z_2) + (t_1 + t_2) \e
\end{equation}

\begin{equation}
(z_1 + t_1 \e) (z_2 + t_2 \e) = (z_1 z_2) + (t_1 z_2 + z_1 t_2) \e
\end{equation}

Notice how with the dual-imaginary unit $z_2 t_2 \e^2$ vanish into oblivion when with the complex-imaginary unit it becomes real.

Before talking about division, we can single out the set $\A = \{0 + t \e | t \in \C\}$ of purely imaginary-dual numbers. In fact, division is a bit problematic. We can observe three cases when dividing $w_1 = z_1 + t_1 \e$ by $w_2 = z_2 + t_2 \e$:

\begin{enumerate}
        \item $z_2 \neq 0$. In this case division is well defined and given by
        \begin{equation}
        \frac{z_1 + t_1 \e}{z_2 + t_2 \e} = \frac{(z_1 + t_1 \e)(z_2 - t_2 \e)}{(z_2 + t_2 \e)(z_2 - t_2 \e)} = \frac{z_1 z_2 - z_1 t_2 \e + t_1 z_2 \e}{z_2^2} = \frac{z_1}{z_2} + \frac{z_2 t_1 - z_1 t_2 }{z_2^2} \e
        \end{equation}
        \item $z_2 = 0$ and we divide the dual-imaginary part ($z_1 = 0$). The usual constraint that $w' = \frac{w_1}{w2} \iff w'w_2 = w1$ leaves us (infinitely) many solutions. Let's say that we want to find $w' = z' + t'$, the quotient of $w1$ and $w2$.
        \begin{equation}
        z' + t' \e = \frac{t_1 \e}{t_2 \e} \iff z' t_2 \e + t' t_2 \e^2 = t_1 \e \iff z' = t_1/t_2
        \end{equation}
        There is no constraint on $t'$. However, a natural definition would be $t' = 0$, implying $\frac{t'_1 \e}{t'_2 \e} = \frac{t'_1}{t'_2}$.
        \item $z_2 = 0$ and we divide the (non-zero) non-dual part $z_1$. In this case we have no solution because $z' + t'\e = \frac{z_1}{t_2 \e} \iff z' t_2 \e = z_1$ which is impossible (by hypothesis $z', t_2, z_1 \in \C$).
\end{enumerate}

Division between $w_1$ and $w_2$ is therefore defined for $w_1 \in \A$ or $w_2 \in \DC \setminus \A$.

Other operations and functions deserve our attention. In particular rooting and exponentiating can be derived using the binomial theorem

\begin{equation}
(z + t\e)^n = z^n + n z^{n-1} t \e
\end{equation}

\begin{equation}
\sqrt[n]{z + t \e} = \sqrt[n]{z} + (\frac{t}{n(\sqrt[n]{z})^{n-1}}) \e
\end{equation}

For the special case where $n = 2$ we have the useful relations $(z+t\e)^2 = z^2 + 2zt\e$ and $\sqrt{z+t\e} = \sqrt{z} + t\e/(2 \sqrt{z})$. Notice how neatly the dual part of the antecedant never seem to appear in the non-dual part of the image in the operation we've presented. More generally if a function $f: \C \to \C$ is analytic, we can extend it to $\DC$ by defining $\hat{f}: \DC \to \DC: z + t\e \to \sum_{n=0}^{\infty} \frac{b^n \e^n f^{(n)}(z)}{n!} = f(z) + b f(z) \e$. A corollary is that we can differentiate $f$ using the formula $f'(z) = (f(z + \e) - f(z))/\e$.

\subsection{Automatic differentiation}

Dual numbers are useful for automatic differentiation. Given an analytic complex-to-complex function $f(x)$:

\begin{equation}
        \hat{f}(x + \e) = f(x) + \e f'(x) \implies f'(x) = (\hat{f}(x) - f(x))/\e
\end{equation}

In numerical analysis or in algebraic computing, we can find $f'(x)$ by implementing $\e$. An efficient way to do so is to use matrices to represent dual numbers. A number $z + t \e$ is represented as

$$
\begin{bmatrix}
z & t \\
0 & z
\end{bmatrix}
$$

Note we can also represent traditional complex numbers as two-by-two matrices, in which case numbers in $\DC$ are represented as four-by-four matrices. This technique is compatible with working with $f$ when it is a function from scalars to vectors or with composing functions.

\subsection{Ordering dual and dual-complex numbers}

We know that while $\R$ can be ordered consistently with the field operations, this is not the case for $\C$ because if $i > 0$ then $i^2 = -1 > 0$. On the other hand if $-i > 0$ then $(-i)^2 = -1 > 0$. Partial orderings can still be defined though.

Luckily this is not the case for (non-complex) dual. There is only one total ordering possible, up to the sign of $\e$. We take $\e > 0$ (choosing $\e < 0$ actually gives the same ordering but using the alternative imaginary unit $\e' = -\e$).

\begin{equation}
\e > r > 0 \implies \e^2 = 0 > r^2 > 0 \text{ is a contradiction and therefore } \e < r \quad \forall r \in R^+
\end{equation}

Intuitively this shows us that we think of $\e$ as an infinitesimal whose square is precisely zero. It also tells us that we can think of the ordering as lexical, i.e.

\begin{equation}
a_1 + b_1 \e \leq a_2 + b_2 \e \iff a_1 < a_2 \lor (a_1 = a_2 \land b_1 \leq b_2)
\end{equation}

As pointed out by Dogukan, we might want to think of $\e$ as another dimension rather than an infinitesimal. We say that $a + b \e \prec c + d \e$ if $a < c$ and $b < d$. This ordering is partial.

\subsection{Conjugation}

The complex conjugate of a complex $z = a + bi$ is defined as $\bar{z} = a - bi$. The same definition is used for (real) dual numbers $w = a + b\e$ ($a, b \in \R$) where the dual conjugate is defined as $\til{w} = a - b \e$. Five definitions of a conjugate are suggested in \cite{messelmi2015}:

\begin{equation}
\bar{w} = \bar{z} + \bar{t}\e
\end{equation}
\begin{equation}
\til{w} = z - t\e
\end{equation}
\begin{equation}
\bar{\til{w}} = \bar{z} - \bar{t}\e
\end{equation}
\begin{equation}
w^* = \bar{z}(1-t\e/z)
\end{equation}
\begin{equation}
w` = t - z \e
\end{equation}

Together with the identity conjugate $id(w) = w$, $\bar{\cdot}$, $\til{\cdot}$ and $\bar{\til{\cdot}}$ form the Klein group. $w^*$ is the only possible definition with both $Re(w) = Re(w^*)$ and $w^*w \in \R$, but is non-linear. This makes it impossible to extend postulates of quantum mechanics, which are expressed with linear algebra. For instance, we lose properties such as $(A + B)^\dagger = A^\dagger + B^\dagger$ making it impossible to use the useful definition $A^\dagger = (A^*)^T$. Other definitions such as $|z| + 2Re(z\bar{t})\e/|z|$ suffer the same limitations. We need therefore a conjugate that is both linear and allows us to define a non-complex norm. $\bar{w}$ is a good candidate.

We have

\begin{equation}
\bar{w_1 + w_2} = \bar{w_1} + \bar{w_2}
\end{equation}

\begin{equation}
\bar{w_1  w_2} = \bar{w_1} ~ \bar{w_2}
\end{equation}

\begin{equation}
\bar{\bar{w}} = w
\end{equation}

\begin{equation}
w + \bar{w} = 2 Re(z) + 2 Re(t) \e
\end{equation}

\begin{equation}
\bar{w}w = |z|^2 + 2 Re(z\bar{t}) \in \D
\end{equation}

\begin{equation}
|w| = \sqrt{\bar{w}w} = |z| + Re(z\bar{t})/|z| \geq 0
\end{equation}

\begin{equation}
|w| = 0 \iff z = 0
\end{equation}

We can then define the adjoint operator $\dagger$ as $A^\dagger = (A^*)^T$ and the norm $||\ket{\psi}||$. In particular we have $A^{\dagger \dagger} = A$, $(AB)^\dagger = B^\dagger A^\dagger$ and $||\ket{\psi}|| \geq 0$.

\section{Extending postulates}

\paragraph{Postulate 1.} Traditional quantum mechanics starts by defining a state as a unit complex vector in a Hilbert space. In our version of QM, entries of vectors are complex duals and the norm is dual. Our task is to define a set of "states" then extend the other postulates to work with this new definition.

Here we decide that we are going to extend functions representing the system over time. Say $\ket{\psi(t)}$ is the state at time $t$. Then we will work with

\begin{equation}
\ket{\hat{\psi}(t + t' \e)} = \ket{\psi(t)} + t' \e \frac{d\ket{\psi(t)}}{dt}
\end{equation}

More generally if we have a complex-state parametrized by a real or complex parameter $z$, we get a complex-dual-"state" when extending $z$ to $\DC$ ($z + t\e$). The non-dual part of this complex-dual-"state" is a complex-state, while the non-dual part can be anything (it depends on the function we are considering). This motivate the following definition of a state:

\begin{equation}
\ket{\psi} \text{ is a a state if and only if } ND(|| \ket{\psi} ||) = 1
\end{equation}

\paragraph{Postulate 2.} Given our definition of a state, we must extend the concept of unitary.

A unitary is a square matrix $U$ such that $||U \ket{\psi}|| = ||\ket{\psi}||$ for all $\ket{\phi}$, so it preserves "stateness". Now if $\ket{\psi_\e}$ is a complex-dual-state and $U_{\e} = U + V \e$ where $U$ is unitary, we quickly see that $U_{\e} \ket{\psi_\e}$ is also a state.

\begin{equation}
||U_{\e} \ket{\psi_\e}||^2 = \bra{\psi_\e} (U + V \e)^\dagger (U + V \e) \ket{\psi_\e} = \braket{\psi_\e | \psi_\e} + O(\e)
\end{equation}

Given our definition of square root, we have that norm $1 + O(\e)$ is preserved. Therefore,

\begin{equation}
U_{\e} = U + V \e \text{ describes the evolution of a closed system if an only if } U \text{ is a unitary. }
\end{equation}

That being said, let us forget about complex-dual unitaries for a moment. Let's say that our system a time $t$ is described by $\ket{\psi(t)}$. Schrödinger's equation says that

\begin{equation}
i\hbar \frac{d \ket{\psi(t)}}{dt} = H \ket{\psi(t)}
\end{equation}

The evolution of our closed system from time $t_1$ to time $t_2$ is described by $U(t_1, t_2) = \exp(\frac{-i H (t_2 - t_1)}{\hbar})$. Note how $H$ and $U(t_1, t_2)$ commute: since $H = \sum_E E \ket{E} \bra{E}$ we have

\begin{equation}
U(t_1, t_2) = \exp(\frac{-i H (t_2 - t_1)}{\hbar}) = \exp(\sum_E \frac{-i E (t_2 - t_1)}{\hbar} \ket{E} \bra{E}) = \exp(\sum_E \frac{-i E (t_2 - t_1)}{\hbar}) \ket{E} \bra{E}.
\end{equation}

$H$ and $U(t_1, t_2)$ are therefore simultaneously diagonalizable and therefore their commutator is zero. Until now, our system is non-dual, so let's add some $\e$. Say we have

\begin{equation}
\ket{\psi_\e(t)} := \ket{\hat{\psi}(t + \e)} = \ket{\psi(t)} + \e \ket{\psi'(t)} = (I + \frac{\e H}{i \hbar}) \ket{\psi(t)}
\end{equation}

What happens when we try to apply $U(t_1, t_2)$ to $\ket{\psi_\e(t)}$? We get

\begin{equation}
U(t_1, t_2) \ket{\psi_\e(t_1)} = (U(t_1, t_2) I + U(t_1, t_2) \frac{\e H}{i \hbar}) \ket{\psi(t_1)} \\
                               & (I U(t_1, t_2) + \frac{\e H}{i \hbar} U(t_1, t_2)) \ket{\psi(t_1)} \\
                               & (I + \frac{\e H}{i \hbar}) U(t_1, t_2) \ket{\psi(t_1)} \\
                               & (I + \frac{\e H}{i \hbar}) \ket{\psi(t_2)} = \ket{\psi_\e(t_2)} \\
\end{equation}

Alternatively, we can directly see this result from what we said in section 1.2. Still this result shows us that keeping

\paragraph{Postulate 3.} A measurement is defined by a collection $\{M_m\}$ of measurement operators with $\sum_m M_m^\dagger M_m = I$. Again, we have the property that $M_m^\dagger M_m$ is semipositive. Given $M_{\e} = M + N \e$

\begin{equation}
M_{\e}^\dagger M_{\e} = M^\dagger M + \e[M^\dagger N + (M^\dagger N)^\dagger]
\end{equation}

$M^\dagger M$ is semipositive and $\bra{\psi} (M^\dagger N + (M^\dagger N)^\dagger) \ket{\psi}$ is always dual (non-complex). In the case where $\bra{\psi} M^\dagger M \ket{\psi} = 0$ we necessarily have that $M \ket{\psi} = \ket{v} \e \in \A^n$ (that is a purely dual-imaginary vector) which means that

\begin{equation}
\e \bra{\psi} (M^\dagger N + N^\dagger M) \ket{\psi} = \e [\e \bra{v} N \ket{\psi} + \bra{\psi} N^\dagger \ket{v} \e] = 0
\end{equation}

Therefore $M_{\e}^\dagger M_{\e}$ is always semipositive. Probabilities are therefore non-negative and sum to $1 + O(\e)$.

\paragraph{Postulate 4.} Suppose $\ket{\psi_1}$ and $\ket{\psi_2}$ are states. Then $\ket{\psi_1} \otimes \ket{\psi_2}$ is also a state. Tensor product is compatible with the other postulates.

\subsection{Other constraints}

\paragraph{Partial ordering} As suggested by Dogukan, rather than considering $\e$ an infinitesimal number, we may be simply thinking of it as another dimension. We consider the partial ordering $\prec$. In this case, not all operators can be considered valid for a measurement, because we still want nonnegative probabilities. Enforcing $M_{\e} = M + \e N$ such that $M^\dagger N \geq 0$ is enough to guarantee that both the dual and non-dual part of the probability is non-negative.

\paragraph{Positive extension} We may require the norm of states to be $1 + b \e$ with $b \geq 0$ (we extend the Bloch sphere in only one direction). In this case we have

\begin{equation}
||U_{\e} \ket{\psi}||^2 = \bra{\psi} (U + V \e)^\dagger (U + V \e) \ket{\psi} = \braket{\psi | \psi} + \e \bra{\psi}(U^\dagger V + V^\dagger U) \ket{\psi}
\end{equation}

Enforcing a similar constraint that $U^\dagger V \geq 0$ is suffient to maintain a valid norm.

\paragraph{Norm is exactly one} Rather than allow a norm of one plus a dual-imaginary part, we can enforce a norm of precisely one.
Evolution must conserve this property so $U^\dagger V + V^\dagger U = 0$.
It is still possible to get dual probabilities when measuring, even when enforcing that the sum of measurement operators is the identitiy.

A vector
\begin{equation}
\ket{\psi} = \braket{\psi_1, \dots, \psi_n} = \braket{z_1 + t_1 \e, \dots, z_n + t_n \e} = \braket{a_1 + b_1i + (c_1 + d_1i)\e, \dots, a_n + b_ni + (c_n + d_ni)\e}
\end{equation}

is real if and only if

\begin{equation}
\sum_i Re(z_i \bar{t_i}) = \sum_i \bar{z_i} t_i + z_i \bar{t_i} = \sum_i a_ic_i + b_id_i = 0
\end{equation}

\bibliographystyle{johd}
\bibliography{bib}

\end{document}